\documentclass[a4j,twocolumn]{jsarticle}
\usepackage[dvipdfmx]{graphicx}
\usepackage{url}

\setlength{\textheight}{275mm}
\headheight 5mm
\topmargin -30mm
\textwidth 185mm
\oddsidemargin -15mm
\evensidemargin -15mm
\pagestyle{empty}


\begin{document}

\title{my\_helpのスマートフォンからの閲覧機能の追加}
\author{情報科学科 \hspace{5mm} 27020708 \hspace{5mm} 鈴木 博大}
\date{}

\maketitle
\label{sec:org5a1197a}
西谷研では,メモソフトとしてmy\_helpを使用している.
メモは,頻繁に読み返すことが重要であるが,
my\_helpではPCを起動させる必要がある.
手軽にメモを読み返すためには,
スマートフォンやタブレットなどPC以外の端末からでも,
メモの閲覧が可能となる必要がある.
そこで本研究では,
my\_helpで作成されるorgファイルの
スマートフォンからの閲覧機能の追加を目的とした.

\section{手法}
\label{sec:orge37fc15}
my\_helpでの実装を考えると,
図\ref{fig:orgce283a0} の通り多様な
オペレーティングシステム(OS)と
クラウドストレージの組み合わせを想定している.
my\_helpではCUIでの動作が前提であるため,
GUIを極力使用しない実装を考える.
今回は,実線で示した
WindowsとGoogleDrive間のデータ共有での設定を行った.
ここで,Windowsは
Windows11上のWSL(windows subsystem for linux)上の
Ubuntuでの動作を考えて,Linux標準コマンドの
mountを試した.
これにより,様々なクラウドストレージへアクセスし,
ファイルやディレクトリの同期を行うことが可能と考えられる.

\begin{figure}[htbp]
\centering
\includegraphics[width=9cm]{./../figs/plat_cloud.png}
\caption{\label{fig:orgce283a0}オペレーティングシステム(OS)とクラウドストレージ間の組み合わせ.}
\end{figure}


\section{結果}
\label{sec:org220798e}
my\_helpで作成したorg形式のメモを
スマートフォンから閲覧できるようにする手順は以下の通りである.
\begin{enumerate}
\item orgファイルをhtmlに変換するために,Emacs上で
orgmode-to-html(C-c C-e h h)を実行.
\item Linux環境下で新しいディレクトリ'/mnt/g'を作成するために
'sudo mkdir /mnt/g'のコマンドを実行.
\item Google Driveのデスクトップを起動すると
Gドライブとして認識される\cite{mount} .
\item 'sudo mount -t drvfs G: /mnt/g'を実行して,マウント.
\item 'cp FILE.html /mnt/g/my\_drive'
で'FILE.html'をGドライブの'my\_drive'にコピー.
\end{enumerate}
この手順は,
'mount'コマンドを使用して仮想的に
ディレクトリに接続するものである.
ここで'drvfs'はWSL上から
マウントされた外部を管理するファイルシステムであり,
'-t'はファイルシステムのタイプを指定するコマンドである.

図\ref{fig:org9c2d906}はメモファイルを,
htmlに表示形式を変えてスマートフォン上で
表示させた画像である.
目次が示され,そのリンクを辿ることで,
見返したい内容を瞬時に表示できるようになった.

\begin{figure}[htbp]
\centering
\includegraphics[width=4.5cm]{./../figs/sumaho1.png}
\caption{\label{fig:org9c2d906}メモをスマートフォンから表示.}
\end{figure}

\section{今後}
\label{sec:orgbdc123f}
この一連の操作をコマンドで動かせるようにし,
my\_helpの中に組み込みむ.
そこで,どの手順を自動化するか,
どのような状況下でこれらのコマンドを
使用するかを検討していく.
また,WSLだけではなく,
OSX,linuxなど他のOSではどうするか.
異なるクラウドの場合の違った手段やコマンドを調べ,
my\_helpの機能として組み込む.


\small\setlength\baselineskip{10pt}
\begin{thebibliography}{9}
\bibitem{mount} Linuxのマウント(mount)について, "mount", https://eng-entrance.com/linux-mount.
\bibitem{drvfs} WSL でマウントしたファイルシステムでもパーミッションを扱えるようにする, "drvfs", https://atmarkit.itmedia.co.jp/ait/articles/1806/08/news042.html.
\end{thebibliography}
\end{document}